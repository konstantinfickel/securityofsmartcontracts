After Bitcoin, the second most popular crypto currency\footnote{according to marked capitalization, see \cite{coinmarketcap:overview}} is \textit{Ether}, which is powered by the \textit{Ethereum Blockchain}. While Ether is the most common denomination of the crypto currency, the values are stored as integers in the unit Wei, which amounts to \( 10^{-18} \textnormal{ Ether}\).

In contrast to Bitcoin, being the ledger recording the simple transfers of the crypto currency is not the main purpose of the Ethereum blockchain: Ethereum was planned to be an general-purpose execution platform for smart contracts, as described in its Whitepaper (\cite{ethereum:wiki}}) that was published in 2014 by the Russian-Canadian programmer Vitalik Buterin.

The blockchain was then launched in 2015, and is maintained by the Ethereum Foundation that is based in Switzerland. They provide reference implementations of the client and maintain descriptions of the protocol and its changes, which the miners can choose to adapt.

In the following chapter, an introduction to the concepts relevant for understanding vulnerabilities of smart contracts on the Ethereum blockchain will be provided.
