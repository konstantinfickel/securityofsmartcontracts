\subsection{Storage Alignment}
\label{section:deepdive:memory}
From the perspective of a smart contract programmer, the persistent smart contract storage can be seen as an monstrous array containing \( 2^{256} \) elements, which is the amount of possible values a \mintinline{solidity}{uint}, whose elements initially are all filled with zeroes. Every element of the storage array can be modified and changed arbitrarily by the programmer.

Solidity uses the following technique to map variables to storage cells: The first state variable declared inside the contract is saved in slot \( 0 \), the next variable follows respectively in the following slot. If a storage variable like a \mintinline{solidity}{struct} or a fixed-size array (for example \mintinline{solidity}{uint256[]}) requires more than one memory slot, the subsequent slot is used.

Dynamically sized state variables like arrays (\mintinline{solidity}{uint[]}) or mappings (\mintinline{solidity}{mapping(string -> address)}) are saved in a different way: At the beginning of the array, only one slot is used. In there, the length will be saved, if the element is a dynamic array, otherwise the cell will retain value \mintinline{solidity}{0}.

The location of the array elements is determined by hashing the position in the array or the key in the mapping, in short by hashing \mintinline{solidity}{uint256(keccak256(key, slot))}. This memory layout relies on the assumption, that no collisions of the \texttt{keccak256}-algorithm will be found.\footnote{see \cite{programmingtheblockchain:storagelayout} and \cite[Layout of State Variables in Storage]{ethereum:solidity}}
