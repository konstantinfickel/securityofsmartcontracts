The state of the Ethereum blockchain consists of accounts, that can either be controlled by \textit{external actors} using public-key cryptography or by smart contracts. For interaction between accounts, \textit{messages} between accounts are launched, which can create new contracts, transfer Ether or call functions of smart contracts. The Ethereum blockchain is lifeless, to start an execution a \textit{transaction}, a message by an external actor, is required. This can lead to \textit{internal transactions}, messages initiated by smart contracts.

Since smart contracts are validated by every node of the network by simply checking if their execution has the same result, smart contracts on Ethereum have to be deterministic.

\subsection{Ethereum Virtual Machine}
\label{section:deepdive:evm}

To formally describe smart contracts, the instructions are encoded as a \textit{bytecode} for a stack machine, which is called the \textit{Ethereum Virtual Machine} (short: EVM). There are around 140 different instructions, ranging from basic arithmetical and logical operations on the stack items, possibilities to fetch information about the blockchain state, stack management like swapping or pushing items, and contract storage management up to ways to interact with other smart contracts.

In an contract execution, aside from the stack, three different kinds of memory can be used: The \textit{calldata} contains information by the sender of current message and can only be read; a non-persistent \textit{memory} that can be accessed arbitrarily, and a contract \textit{storage} that is persistent over the lifetime of the contract and therefore is part of the blockchain state.

Since the Ethereum Virtual Machine is Turing-Complete, there is no way to predict whether the execution of a smart contract will terminate. Because of this, the so-called \textit{gas-system} has been implemented, whose goal is to limit the resources required to execute (and therefore validate) the transactions. Gas is a fee, that users have to pay to the miner of the block for every operation executed during their transaction. The amount of gas required by the transaction is determined by the computational resources required for each instruction; and the \textit{gas price} can be dynamically set by the initiator of the transaction. If the gas price is set too low the transaction will be only executed after a long time or will even never be included into a block.\footnote{see \cite[Whitepaper]{ethereum:wiki}}

At the beginning of every transaction, the maximum amount of gas allowed to be used during its execution has to be set, which is called \texttt{STARTGAS}. Every instruction for the Ethereum has a gas price, which is supposed to be proportional to the resources required for execution. In every computation step, the gas price of the current instruction is deducted from the \texttt{STARTGAS}. If at the end of the execution there is gas left, the remaining gas value is refunded to the user that started the transaction. If a computation runs out of gas or fails due to an error, the whole transaction is reverted and the gas is left for the miner.

Using the designated \texttt{CALL} operation, a contract can send messages to other accounts, for example to retrieve data from other contracts or to send Ether. This new message includes among others the address of the called contract, the Ether sent along with the message, the message data and the gas that can be used by the sub-execution has to be specified, of which the later can limit the operations that can be done by the callee.

The price paid per gas unit to the miners can be set by the creator of the transaction. Due to higher revenue, the miners are incentivized to include the transactions with the highest gas price -- making its choice a tradeoff between the costs and the time until the transaction is included into the block.\footnote{see \cite[Whitepaper]{ethereum:wiki}}
