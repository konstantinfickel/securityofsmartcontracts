\subsection{Solidity}
Since smart contract bytecode (even in the form of assembler) is difficult to read, there are several high-level languages for smart contract development. The most popular language of those is Solidity, which has a syntax similar to JavaScript, but with static types. This makes smart contracts simple to be used for web programmers, but sets wrong expectations for the final completely-public and resource-bounded execution in the EVM.

Solidity is a contract-oriented language: Instead of classes as in object-oriented languages, the code is structured into smart contracts, which will later compile to the code that can be deployed to a smart contract account.

In listing \ref{lst:introduction:caution} a smart contract version of an deposit account is presented, that is meant to be a protection for a landlord when renting out a flat. The smart contract is an agreement between the landlord and the tenant, that ensures that the tenant is only able to withdraw the caution, if the landlord allows the withdrawal (for example after the tenancy has been quit).

\begin{listing}
	\begin{minted}[
        linenos=true,
        firstnumber=1
    ]{solidity}
pragma solidity ^0.4.24;

contract Caution {
    address public tenant;
    address public landlord;
    bool public landlordAllowsWithdrawal = false;

    constructor(address _landlord) public payable {
        landlord = _landlord;
        tenant = msg.sender;
    }

    function allowWithdrawal(bool _newDecision) public {
        require(msg.sender == landlord);
        landlordAllowsWithdrawal = _newDecision;
    }
    
    function withdraw() public {
        require(msg.sender == tenant && landlordAllowsWithdrawal);
        selfdestruct(tenant);
    }
}
    \end{minted}
	\caption{A fully-implemented Caution smart contract}
	\label{lst:introduction:caution}
\end{listing}

The contract contains two different types of entries: There are state variables like \mintinline{solidity}{address public tenant} that will be stored in the persistent contract storage. The other part of the contract are functions, that will be compiled to the code of the smart contracts. A special function of the constructor, that is either a function with the same name as the contract, or a function without identifier using the \mintinline{solidity}{constructor} keyword. The code inside the constructor will be executed during contract creation, and will later not become a part of the contract code in the blockchain state.

If functions or state variables of the smart contracts are meant to be accessible for other smart contracts on the blockchain, they can be marked as \mintinline{solidity}{public} and will therefore become part of the contract interface, which will be explained in the next section.\footnote{see \cite{ethereum:solidity}}
