\section{Outlook}
When considering the frequency and simplicity of most attacks against Ethereum smart contracts, it is very likely that there are still a lot of large-scale vulnerabilities to be found in the future. Additionally, there are many other challenges regarding smart contracts aside from security issues that need to be solved:

First of all, there are problems with the underlying blockchain technology. Since most blockchains still rely on a proof of work algorithm, it consumes a lot of energy; for the Ethereum the energy consumption is larger than the energy consumption of Iceland as a country.\footnote{as stated on \cite{digiconomist:ethereumenergy}} Furthermore, since every full node in the network has to validate every newly included transaction, the amount of transactions is heavily limited to still allow regular computers to participate in the network: While Visa is able to handle more than 1.500 transactions per second, the Ethereum blockchain is limited to about 20 transactions per second.\footnote{as stated by \cite{altcointoday:paypalvsethereum}} As Matzutt et al. found out in their paper \cite{matzutt:arbitraryblockchaincontent}, the bitcoin blockchain includes child pornography which can not be removed due to the append-only properties; thus making the possession of the complete blockchain data illegal. The Ethereum blockchain might be facing similar problems, since it is even easier to store arbitrary content there.

Also, there are legality concerns for smart contracts, which will be explained as in this section at the example of Germany: It is completely legal to write agreements in the form of programs instead of natural language, but if due to programming errors the contract works different from the intention, or even is hacked due to vulnerabilities, the effects would have to be reverted. Additionally, when considering smart contracts as contracts in the context of law, it becomes illegal for programmers to offer their skills for smart contract programming, since in Germany law consulting is only allowed to lawyers.

Another problem occurs when trying to use smart contracts to watch functionalities outside the blockchain: The get information about the real world, the blockchain requires oracles running outside the blockchain providing smart contracts with data, which for most cases are difficult to program, whose correct execution is even more difficult to be verified.\footnote{see \cite{ct:smartcontracts} and \cite[page 17]{vbwbayern:blockchain}}

But still, due to the inventions of blockchains, it was finally possible for Nick Szabos idea from 1997 to become real. And using newly invented ideas for application and progresses in research for improved security, there is a good change that in a few years they will become a regular part of the business world.

\pagebreak{}
