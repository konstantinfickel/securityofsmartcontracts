\section{Using Sact}
\label{appendix:abicollision}
In this section, the a short usage guide for the \textit{Solidity ABI Collision Tool} (or \textsc{Sact}) will be presented. The dependencies for building the project are \texttt{gcc} and \texttt{make} along with the libraries \texttt{libkeccak}, \texttt{argp} and \texttt{uthash}; the build process can be launched by calling \texttt{make} inside the folder.

Using \texttt{./collision --help} the integrated help page with the following contents can be displayed:
\begin{verbatim}
$ ./collision --help
Usage: collision [OPTION...] from to
A program calculating collisions of the Solidity ABI.

  -c, --threads=INTEGER      Amount of threads used when no dictionary is used
                             at the from-side.
  -f, --from-prefix=STRING   Prefix to be used on the from-side.
  -m, --mode=HASH|DICTIONARY|NONCE
                             Mode of execution.
  -t, --to-prefix=FILE       Prefix to be used on the to-side.
  -?, --help                 Give this help list
      --usage                Give a short usage message

Mandatory or optional arguments to long options are also mandatory or optional
for any corresponding short options.
\end{verbatim}

\subsection{Calculating Hashes}
To calculate the function hash identifier of a fixed function, the function identifier can be entered as the first parameter -- and the hash value is printed out to the console.

\begin{verbatim}
$ ./collision "find my hash"
9bb03d0f
\end{verbatim}

\subsection{Finding Collisions}
The primary application of \textsc{Sact} is finding hash identifier collisions. To do this, the tool includes two different modes:

\subsubsection{One-sided Search}
In this first mode activated by \texttt{--mode=NONCE}, a function identifier with a matching hash value is attempted to be found for a fixed target function identifier. The tested identifiers are built using the following formula:
\begin{verbatim}
[from-prefix][nonce in the form of [A-Za-z]+][from]
\end{verbatim}

Since without any optimization this would take about an hour, the tool is able to use multiple threads; their amount can be defined using for example \texttt{--threads=4} for four threads. This mode is the only mode using more than one thread, for all other modes the argument will be ignored.

\begin{verbatim}
$ ./collision --mode=NONCE --threads=4 "()" "executeSomething()"
[Thread 01] hash(A()) == f446c1d0 != d90fec61 == hash(executeSomething())
[Thread 02] hash(FPXc()) == 3796f4d4 != d90fec61 == hash(executeSomething())
[Thread 03] hash(KeuEB()) == 0412673b != d90fec61 == hash(executeSomething())
[Thread 04] hash(PtRhB()) == 794e29e8 != d90fec61 == hash(executeSomething())
...
hash(wnpWTF()) == hash(executeSomething()) == d90fec61
\end{verbatim}

If a function identifier has been found, the tool terminates printing out the final solution.

This functionality can be used to send arbitrary data to other contracts in restricted test environments like IDEs; or for obfuscation of the interaction with other contracts.

\subsubsection{Two-sided Search}
In the second mode of \textsc{Sact}, more natural-looking identifiers can be created for hash-collisions. To define possible identifers, so-called \textit{dictionary}-files can be used. A dictionary contains an ordered list of \textit{segments}, which are a set of multiple ore one strings. To generate an identifier, from each segment one string is taken and concatenated in order.

Dictionaries can be stored in dictionary files, which are simple text files. The newline-character is used as a delimiter for the strings; with the special strings \texttt{[NEXT]} starting the next segment of the dictionary and \texttt{[END]} ending the read input. The later command leaves the possibility for comments or information required by the text editor. The following dictionary file would exactly generate the two entries \texttt{"Hello World!"} and \texttt{"Hello C!"}.

\begin{verbatim}
Hello
[NEXT]
World
C
[NEXT]
!
[END]
\end{verbatim}

Two find prettier hash collisions, it would be too difficult to build up a dictionary with more \( 2^{32} \) entries; it is simpler to use two different dictionaries with \( \sqrt{2^{32}} = 2^{16} \) terms. Because of this, this mode is built to use dictionaries both as source and target of the collision.

The dictionary mode can be launched using \texttt{--mode=DICTIONARY}, specifying both used dictionary files as the arguments:
\begin{verbatim}
$ ./collision "data/from.txt" "data/to.txt" --mode=DICTIONARY
hash("transferMyMoneyToTheAddressPlease(uint232,bool,address)")
    = hash("validateInputWealthInstantaneously(int256[])") = 014f3c83
hash("transferMyCashToTheAccount(uint256,bool,address)")
    = hash("validateInputsAndGetSzaboGently(int256[])") = 60c3ba0f
...
\end{verbatim}

Note that, for improved performance and storage usage, the smaller dictionary should be used on the to-side, because this side will be stored in the memory for comparison.

\pagebreak{}
